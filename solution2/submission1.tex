\documentclass[10pt,a4paper]{scrartcl}
\usepackage[utf8]{inputenc}
\usepackage{lmodern}
\usepackage[intlimits]{amsmath}
\usepackage[hidelinks=true]{hyperref}
\usepackage{url}
\usepackage{breakurl}
\usepackage{booktabs}
\usepackage{amsfonts}
\usepackage{amssymb}
\usepackage{graphicx}
\usepackage{todonotes}
\usepackage{enumerate} 
\usepackage{bm} 
\usepackage{physics} 
\usepackage{cleveref} 
\usepackage{pgfplots} 
\usepackage{csquotes} 

\newcommand{\ve}[1]{\bm{#1}} 
\newcommand{\Dv}[2]{\frac{\mathrm{D}#1}{\mathrm{D}#2}} 

\newcounter{problemcounter}

\newenvironment{problem}{%
\refstepcounter{problemcounter}%
\noindent%
\textbf{Problem \theproblemcounter:} }{}

\usepackage[separate-uncertainty=true, exponent-product=\cdot]{siunitx}
\begin{document}
\title{\Huge Submission 2}
\author{Philipp Stärk}
\date{\small \today}
\maketitle
\paragraph{Hinweis}%
\label{par:hinweis}
Ich benötige keinen Schein, ich höre Adv. Stat. Mech. nur aus Interesse.
Meine Abgaben müssen daher nicht korrigiert werden, allerdings möchte ich sie dennoch machen, um mich am Ball zu halten.
\vspace{1em}


\begin{problem}
\begin{enumerate}[a)]
\item Free energy calculation: We know $ S_I = \beta (U_I -F_I) $.
     \begin{gather*}
         F_I = U_I - \beta^{-1} S_I\\
     \implies F_1 = - \ln (2)/\beta\\
     F_2 = E
     \end{gather*}
\item Starting with the definition for the free intrinsic energy $ p_I^e = \exp(-\beta (F_I -F)) $, thus we have:
     \begin{gather*}
         p_1^e = \frac{\exp(\beta \beta^{-1} \ln(2))}{Z}\\
         p_2^2 = \frac{\exp(-\beta E)}{Z},
     \end{gather*}
     where the partition sum $ Z = p_1^e + p_2^e$ is the normalization factor.
     \begin{equation*}
         Z = \sqrt{2 + \exp(-\beta E)}.
     \end{equation*}

\item Starting with the transition rate $ k^+ = K_{12} $ and knowing:
    \begin{equation*}
        \frac{k^+}{k^-} = \exp(-\beta \Delta _{12} F)
    \end{equation*}
    Where $ \Delta _{12}F $ is the energy difference between meso-state 1 and 2.
    Thus we have
    \begin{equation*}
        k^- = \frac{k^+}{\exp(-\beta E + \ln 2)} = \frac{k^+}{\exp(-\beta E) + 2}.
    \end{equation*}

\item The master equations read
    \begin{align*}
        \label{eq:ma_eq}
        \partial_t p_1(t) &= -k^+ p_1(t) + p_2(t)k^-\\
        \partial_t p_1(t) &= k^+ p_1(t) - p_2(t)k^-\\
        \Leftrightarrow:
        \partial_t \ve{p}(t) &= \underbrace{\begin{pmatrix}-1 & 1\\1&-1\end{pmatrix}}_{=:A} \ve{p}(t)
    \end{align*}
    This linear system of differential equations of the first order can be solved by getting the eigenvalues and eigenvectors of $ A $.

    Hence, the general solution is given by:
    \begin{equation*}
        \ve{p}(t) = c_1e^{-2t} \begin{pmatrix}-1\\1 \end{pmatrix} + c_2\cdot \begin{pmatrix}1\\1\end{pmatrix}, c_i \in \mathbb{R}.
    \end{equation*}
    The parameter $ c $ can be determined by inserting the initial condition:
    \begin{equation*}
        \ve{p}(0) \overset{!}{=} \begin{pmatrix}1-\epsilon\\\epsilon\end{pmatrix} = c_1 \begin{pmatrix}-1\\1 \end{pmatrix} + c_2\begin{pmatrix}1\\1\end{pmatrix}.
    \end{equation*}
    Which solves to $ c_1 = \epsilon - 1/2 $ and $ c_2 = 1/2 $ .

\item From the lecture we know
    \begin{align*}
        \expval{\dot{Q}} (t) &= \sum _{I,J} p_I (t) K_{IJ} Q_{IJ} = p_1(t) k^+ Q_{12} + p_2(t) k^- Q_{21}\\
                             &= ((\epsilon - 1/2) e^{-2t} - 1/2 ) k^+E + (\epsilon e^{-2t}+1/2)k^- E\\
        \implies \expval{\dot{Q}} (0) &= ((\underbrace{\epsilon}_{\to 0}-1/2) - 1/2) k^+ E + (\underbrace{\epsilon}_{\to 0} + 1/2)k^- E \\ 
                                      &= -k^+ E + \frac{1}{2}\frac{k^+}{\exp(-\beta E) + 2}E,
    \end{align*}
    the negativity of this quantity stems from the fact that the initial configuration is primed to be in the ground state.
    Thus, overall the system will \enquote{want} to acquire more energy.
    The convention is to denote energy taken from the heat bath (which is what will happen on average) as a negative number.

\item The instantaneous total entropy production rate is determined by
    \begin{equation*}
        \expval{\dot{S}_{\text{tot}}} (t) = \sum_{IJ} \left( p_I K_{IJ} - p_J K_{JI} \right) \ln \frac{p_I K_{IJ}}{p_J K_{JI}} = (p_1 k^+ - p_2 k^-) \ln \frac{p_1 k^+}{p_2 k^-},
    \end{equation*}
    Inserting what we have calculated before, we get
    \begin{gather*}
        \expval{\dot{S} _{\text{tot}}} (t) = \left(k^{+} \left(\left(0.5 - \epsilon\right) e^{- 2 t} + 0.5\right) - \frac{k^{+} \left(\left(\epsilon - 0.5\right) e^{- 2 t} + 0.5\right)}{2 + e^{- \beta E}}\right)\cdot\\ 
        \log{\left(\frac{\left(2 + e^{- \beta E}\right) \left(\left(0.5 - \epsilon\right) e^{- 2 t} + 0.5\right)}{\left(\epsilon - 0.5\right) e^{- 2 t} + 0.5} \right)}.
    \end{gather*}

\item Trajectory with $ \Delta S _{\text{tot}} [I(t)] < 0 $.
    Let's start with the definition for the total entropy production for a given trajectory:
    \begin{gather*}
        \Delta S _{\text{tot}}[I(t)] = \ln \frac{p [I(t)]}{p[\tilde{I}(t)]} = \ln \frac{p[I(t)|I^0]}{p[\tilde{I}(t)|I^T]},
    \end{gather*}
    where $ p[I(t)|I^0] = \left[\prod_{l=1}^L e^{-\Gamma _{I_l} (t_l - t _{l-1})} K_{I_l^- I_l^+}\right]e^{-\Gamma_{I^T}(T-t_l)} $.
    Imagine a trajectory $ I(t) $ which starts in state 2 and at time $ \tau $ jumps to state 1, remaining there till $ T $.
    Then, we have:
    \begin{align*}
        \label{eq:}
        p[I(t)|I^0] &= e^{-k^- \tau} k^- e^{-k^+ (T-\tau)}\\
        p[\tilde{I}(t)|I^T] &= e^{-k^- \tau} k^+ e^{-k^+ (T-\tau)}\\
        \implies \frac{p[I(t)|I^0]}{p[\tilde{I}(t)|I^T]} &= \frac{k^-}{k^+} = \frac{1}{\exp(-\beta E) + 2} =: B,\\
        \implies \ln B &< 0.
    \end{align*}
\end{enumerate}
\end{problem}

\end{document}
