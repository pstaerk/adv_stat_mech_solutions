\documentclass[10pt,a4paper]{scrartcl}
\usepackage[utf8]{inputenc}
\usepackage{lmodern}
\usepackage[intlimits]{amsmath}
\usepackage[hidelinks=true]{hyperref}
\usepackage{url}
\usepackage{breakurl}
\usepackage{booktabs}
\usepackage{amsfonts}
\usepackage{amssymb}
\usepackage{graphicx}
\usepackage{todonotes}
\usepackage{enumerate} 
\usepackage{bm} 
\usepackage{physics} 
\usepackage{cleveref} 
\usepackage{pgfplots} 

\newcommand{\ve}[1]{\bm{#1}} 
\newcommand{\Dv}[2]{\frac{\mathrm{D}#1}{\mathrm{D}#2}} 

\newcounter{problemcounter}

\newenvironment{problem}{%
\refstepcounter{problemcounter}%
\noindent%
\textbf{Problem \theproblemcounter:} }{}

\usepackage[separate-uncertainty=true, exponent-product=\cdot]{siunitx}
\begin{document}
\title{\Huge Submission 2}
\author{Philipp Stärk}
\date{\small \today}
\maketitle

\begin{problem}
\begin{enumerate}[a)]
\item Free energy calculation: We know $ S_I = \beta (U_I -F_I) $.
     \begin{gather*}
         F_I = U_I - \beta^{-1} S_I\\
     \implies F_1 = - \ln (2)/\beta\\
     F_2 = E
     \end{gather*}
\item Starting with the definition for the free intrinsic energy $ p_I^e = \exp(-\beta (F_I -F)) $, thus we have:
     \begin{gather*}
         p_1^e = \frac{\exp(\beta \beta^{-1} \ln(2))}{Z}\\
         p_2^2 = \frac{\exp(-\beta E)}{Z},
     \end{gather*}
     where the partition sum $ Z = p_1^e + p_2^e$ is the normalization factor.
     \begin{equation*}
         Z = \sqrt{2 + \exp(-\beta E)}.
     \end{equation*}

\item Starting with the transition rate $ k^+ = K_{12} $ and knowing:
    \begin{equation*}
        \frac{k^+}{k^-} = \exp(-\beta \Delta _{12} F)
    \end{equation*}
    Where $ \Delta _{12}F $ is the energy difference between meso-state 1 and 2.
    Thus we have
    \begin{equation*}
        k^- = \frac{k^+}{\exp(-\beta E + \ln 2)} = \frac{k^+}{\exp(-\beta E) + 2}.
    \end{equation*}

\item The master equations read
    \begin{align*}
        \label{eq:ma_eq}
        \partial_t p_1(t) &= -k^+ p_1(t) + p_2(t)k^-\\
        \partial_t p_1(t) &= k^+ p_1(t) - p_2(t)k^-\\
        \Leftrightarrow:
        \partial_t \ve{p}(t) &= \underbrace{\begin{pmatrix}-1 & 1\\1&-1\end{pmatrix}}_{=:A} \ve{p}(t)
    \end{align*}
    This linear system of differential equations of the first order can be solved by getting the eigenvalues and eigenvectors of $ A $.

    Hence, the general solution is given by:
    \begin{equation*}
        \ve{p}(t) = c_1e^{-2t} \begin{pmatrix}-1\\1 \end{pmatrix} + c_2\cdot \begin{pmatrix}1\\1\end{pmatrix}, c_i \in \mathbb{R}.
    \end{equation*}
    The parameter $ c $ can be determined by inserting the initial condition:
    \begin{equation*}
        \ve{p}(0) = \begin{pmatrix}1-\epsilon\\\epsilon\end{pmatrix} = c_1 \begin{pmatrix}-1\\1 \end{pmatrix} + c_2\begin{pmatrix}1\\1\end{pmatrix}.
    \end{equation*}
    Which solves to $ c_1 = \epsilon - 1/2 $ and $ c_2 = 1/2 $ .

\item From the lecture we know
    \begin{equation*}
        \expval{\dot{Q}} (t) = p_I (t) K_{IJ} Q_{IJ}.
    \end{equation*}
\end{enumerate}
\end{problem}

\end{document}
