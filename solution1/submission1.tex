\documentclass[10pt,a4paper]{scrartcl}
\usepackage[utf8]{inputenc}
\usepackage{lmodern}
\usepackage[intlimits]{amsmath}
\usepackage[hidelinks=true]{hyperref}
\usepackage{url}
\usepackage{breakurl}
\usepackage{booktabs}
\usepackage{amsfonts}
\usepackage{amssymb}
\usepackage{graphicx}
\usepackage{todonotes}
\usepackage{enumerate} 
\usepackage{bm} 
\usepackage{physics} 
\usepackage{cleveref} 

\newcommand{\ve}[1]{\bm{#1}} 
\newcommand{\Dv}[2]{\frac{\mathrm{D}#1}{\mathrm{D}#2}} 

\newcounter{problemcounter}

\newenvironment{problem}{%
\refstepcounter{problemcounter}%
\noindent%
\textbf{Problem \theproblemcounter:} }{}

\usepackage[separate-uncertainty=true, exponent-product=\cdot]{siunitx}
\begin{document}
\title{\Huge Submission 1}
\author{Philipp Stärk}
\date{\small \today}
\maketitle

\begin{problem}
Self avoiding polymer model:

\begin{enumerate}[a)]
\item Probability distribution of 4 monomer chain:

Possible values for chain length (found by Drawing): $ |\ve{R}_3 - \ve{R}_0| = 1, \sqrt{5}, 3 $.

Probabilities, determined by enumeration and symmetry arguments:
    \begin{table}[h]
        \centering
        \begin{tabular}{@{}ll@{}}
            \toprule
            $ p $ & Value\\\midrule
            $ p (|\ve{R}_3 - \ve{R}_0| = 1) $ & $\frac{2}{9}$ \\
            $ p (|\ve{R}_3 - \ve{R}_0| = \sqrt{5}) $ & $\frac{6}{9}$ \\
            $ p (|\ve{R}_3 - \ve{R}_0| = 3) $ & $ \frac{1}{9}  $ \\\bottomrule
        \end{tabular} 
    \end{table}

    The probability to find the chain in a straight line is that of $ p(|\ve{R}_3 - \ve{R}_0| = 3) = \frac{1}{9} $.

\item Mean:
    \begin{equation*}
        \expval{\abs{\ve{R}_3 - \ve{R}_0}} = \frac{2}{9} + \sqrt{5} \frac{6}{9} + 3 \frac{1}{9} \approx 2.04
    \end{equation*}

    Variance:
    \begin{equation*}
        \sigma = \sqrt{\expval{\abs{ \ve{R}_3 - \ve{R}_0}^2}} = \sqrt{ \frac{2}{9} + 5 \frac{6}{9} + 9 \frac{1}{9}  } \approx 1.91
    \end{equation*}
\end{enumerate}
\end{problem}

\begin{problem}
Random walk on a lattice without self-avoidance:

\begin{enumerate}[a)]
\item \label{it:a} Probability distribution $ p (X, Y | N) $:

Start with considering random walk in $ x $-direction, assume $ N_x = r + l$ steps in this direction, with $ r $ the steps in positive and $ l $ steps in negative direction, i.e.\  $ X = r - l $.

$ \implies r-l =x, N = r+l \implies r = \frac{N_x + x}{2}, l = \frac{N_x - x}{2} $.

Thus, we have
\begin{equation*}
    p(X|N_x) = \begin{pmatrix}N_x\\\frac{N_x + x}{2}\end{pmatrix} p^{N_x}
\end{equation*}
in one direction (of course analogously in $ y $-direction), with $ p = \frac{1}{2}  $.

To get the probability $ p(X, Y|N) $, we need to sum over all combinations of possible number of path-steps in each direction, where we know:
\begin{gather*}
    N = N_x + N_y,\\
    \forall N_x \in \{ X, \dots,  N-Y \},\\
    \forall N_y \in \{ Y, \dots,  N-X \}.\\
\end{gather*}

Thus, overall, we have:
\begin{equation*}
    p(X, Y|N) = \sum _{N_x = X}^{N-Y} \begin{pmatrix}N_x\\ \frac{N_x + X}{2}\end{pmatrix} \begin{pmatrix}N - N_x\\ \frac{N - N_x + Y}{2}  \end{pmatrix} \frac{1}{4}^N,
\end{equation*}
with the condition that $ X + Y $ is even (odd), if $ N $ is even (odd), otherwise we immediately know $ p(X, Y|N) = 0 $, because there is no possible path to that point.

I am not sure if there is a more elegant, close form solution to this probability distribution, because this is a bit ugly.

\item\label{it:b} Mean value:

    One can reason that there is nothing special about the $ x $-direction, i.e. $ \expval{N_x} = \frac{N}{2} $, thus, we may use the probability distribution $ p(X|N_x) $ as defined above in \ref{it:a}).
\begin{equation*}
    \expval{X|N} = \sum_{X} X p(X|\underbrace{N_x}_{= N/2}) = \sum_{X=0}^{N} X \begin{pmatrix}
        N/2\\ \frac{N/2 + X}{2}
    \end{pmatrix} \frac{1}{2}^{N/2}
\end{equation*}
The same obviously holds for $ \expval{Y|N} $.

\item The path cannot return to (0,0) for odd number of jumps, as $ X + Y = 2\mathbb{Z} $ is only possible for even numbers of steps.

The probability function.

\end{enumerate}
\end{problem}


\end{document}
